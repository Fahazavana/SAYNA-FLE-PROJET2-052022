\documentclass[12pt]{article}
\usepackage[top=10mm,bottom=10mm,left=10mm,right=10mm]{geometry}
\usepackage[T1]{fontenc}
\usepackage[utf8]{inputenc}
\usepackage[french]{babel}
\usepackage{array}
\usepackage{hyperref}
\usepackage{pifont}
\usepackage{colortbl}
\usepackage{color}
\newcommand{\und}[1]{\underline{#1}}
\newcommand{\colo}[1]{{\color{blue}\textbf{#1}}}

\author{Jean Lucien RANDRIANANTENANA}
\title{Vers un français
impeccable (B2)\\SAYNA-FLE-PROJET1-052022}
\begin{document}
\maketitle
\tableofcontents
\newpage
\section{Les connecteurs logiques}
\subsection{Identifiez la relation logique}
\begin{enumerate}
	\item Prenez ce GPS dans le cas où vous vous perdriez. \colo{(Supposition)}
	\item Elle ne parle que de son chien.\colo{Restricrtion}
	\item Je ne portais pas d’imperméable, c'est pourquoi je suis mouillé. \colo{(Conséquence)}
	\item Il fait très chaud depuis le début de l’année, notamment ces derniers jours. \colo{(Illustration)}
	\item J'aime tous les légumes en dehors des navets.\colo{(Restriction)}
	\item Je partirai tôt demain matin. En effet, j'ai beaucoup de travail qui m’attend. \colo{(Justification)}
	\item Il ne parle pas anglais pourtant il est né à Londres. \colo{(Opposition)}
	\item À supposer que vous vouliez venir avec moi, je n'ai pas assez de place dans le van.\colo{(Condition)}
\end{enumerate}
\subsection{INDICATIF OU SUBJONCTIF : Conjuguez les verbes à la forme qui convient}
\begin{enumerate}
	\item Tu es rousse alors que ton frère \colo{est} blond.
	\item Bien qu’il \colo{soit} parti en retard, il est arrivé en avance.
	\item Afin que tu \colo{comprennes} la problématique, je vais te l’expliquer. 
	\item Ahmed \colo{est} plus grand que Fatou.
	\item Sa mère sera rassurée à condition qu'il \colo{suive} des cours particuliers.
	\item J’ai beaucoup étudié, c’est ainsi que je \colo{suis} devenu avocat.
	\item Donne-lui cette casquette qu‘il n’ \colo{ait} pas trop chaud à la tête.
	\item Après que chacun \colo{a} gagné son siège, la séance peut commencer.
\end{enumerate}
\subsection{Choisissez le bon connecteur logique : }
\begin{enumerate}
	\item \colo{Puisque} tu as oublié ton argent, voici 10 euros.
	\item \colo{Comme} il était souffrant, il est retourné chez lui.
	\item Ce pays est très pauvre \colo{en revanche} ses habitants sont très gentils.
	\item J’habite à Antananarivo \colo{depuis} 7 ans.
	\item J’ai voté pour elle \colo{même si} je ne l’apprécie pas.
	\item Il est arrivé régulièrement en retard au travail \colo{par conséquent} il a été renvoyé.
	\item Je vais à la plage cet après-midi, \colo{pourvu qu’} il ne pleuve pas. \item Vous êtes allées dehors \colo{malgré} l’averse.
\end{enumerate}

\subsection{Terminez librement les phrases en utilisant un connecteur logique.}
\begin{enumerate}
	\item Il était absent à la réunion \colo{parce qu'il était malade}
	\item Cette société ne fait pas assez de bénéfice \colo{alors il s'en dette pour payer le salaire des employer}
	\item Elles se sont rendues au marché \colo{pour acheter des légumes}
	\item Vous avez pris le bus \colo{pour aller à la gare puis le train pour aller à Manakara}
	\item Tu n’as pas remporté la victoire, \colo{cependant tu a fait de bon progrès.}
\end{enumerate}

\subsection{Ces connecteurs utilisent-ils l'indicatif ou le subjonctif ?}
\begin{enumerate}
	\item Du fait que : \colo{Indicatif}
	\item Pour que :\colo{Subjonctif}
	\item Alors que  : \colo{Indicatif}
	\item Jusqu'à ce que :\colo{Subjonctif}
	\item Bien que : \colo{Subjonctif}
	\item De sorte que :\colo{Subjonctif}
	\item Après que: \colo{Indicatif}
\end{enumerate}

\section{La condition et l’hypothèse}
\subsection{Conjuguez le verbe au présent (conditionnel, subjonctif ou indicatif)}
\begin{enumerate}

	\item Je vous permets de sortir si vous me \colo{promettez} de revenir dans une heure. (promettre) \\
	      \ding{212} Je vous permets de sortir à condition que vous me \colo{promettiez} de revenir dans une heure.
	\item Si vous \colo{faite} quelques progrès, vos parents seront contents. (faire) \\
	      \ding{212} Pour peu que vous \colo{fassiez} quelques progrès, vos parents seront contents. 
	      \item Venez au concert, si vous ne \colo{préférez} pas vous reposer. (préférer) \\
	      \ding{212} Venez au concert, à moins que vous ne \colo{préfériez} vous reposer.
	\item Si vous \colo{ratez} votre examen, que ferez-vous ? (rater) \\
	      \ding{212} Au cas où vous \colo{ratiez} votre examen, que ferez-vous ? \\ 
	      \ding{212} En admettant que vous \colo{ratiez} votre examen, que ferez-vous ? 
	      \item Vous gagnerez le concours si vous \colo{vous entrainer} (s'entraîner) \\ 
	      \ding{212} Vous gagnerez le concours à condition que vous \colo{vous entraîniez} \\ 
	      \ding{212} Vous gagnerez le concours pour peu que vous \colo{vous entraîniez}
	\item Vous ne vous servirez du dictionnaire que si le professeur vous le \colo{permet} (permettre) \\
	      \ding{212} Vous ne vous servirez pas du dictionnaire, à moins que le professeur vous le \colo{permette}
	\item Tu iras chez le dentiste, si ça te \colo{plaît} ou si ça ne te \colo{plaît} pas. (plaire) \\
	      \ding{212} Tu iras chez le dentiste, que ça te \colo{plaise} ou non.
\end{enumerate}
\subsection{Choisissez la bonne réponse}
\begin{enumerate}
		\item Dès que vous aurez terminé, nous \colo{partirons}\\ 
	      \ding{212} nous partirions \\ 
	      \ding{212} nous partirons 
	      \item Si vous étiez d'accord, nous \colo{commencerions} les travaux demain \\ \ding{212} nous commencerons \\ 
	      \ding{212} nous commencerions 
	      \item Je \colo{vendrai} ma voiture le mois prochain \\ 
	      \ding{212} je vendrai \\ 
	      \ding{212} je vendrais 
	      \item J'\colo{écrirai} bien cette lettre, mais je crains de faire des fautes \\ \ding{212} j'écrirai \\ 
	      \ding{212} j'écrirais 
	      \item Petite annonce : \colo{prendrai} des enfants en garde \\ 
	      \ding{212} (je) prendrai \\ 
	      \ding{212} (je) prendrais 
	      \item Autre annonce : \colo{donnerai} leçons de français et de karaté \\ \ding{212} (je) donnerai \\ 
	      \ding{212} (je) donnerais 
	      \item Je \colo{pourrais} vous aider si je ne devais pas m'absenter \\ \ding{212} je pourrai \\ 
	      \ding{212} je pourrais.
	      \item Je \colo{tacherai} de vous aider dès mon retour \\ 
	      \ding{212} je tâcherai \\ \ding{212} je tâcherais 
	      \item Il \colo{voudra} bien savoir comment faire. \\ 
	      \ding{212} "il (ou elle) voudra \\ 
	      \ding{212} "il (ou elle) voudrait" 
	      \item Je \colo{repondrai} sans faute \\ 
	      \ding{212} répondrai \\ 
	      \ding{212} répondrais
\end{enumerate}
\section{Les pronoms relatifs composés} 
\subsection{Complétez les phrases suivantes en utilisant les pronoms relatifs : lequel, laquelle, lesquels ou lesquelles.} 
\begin{enumerate}
\item Les raisons (f) pour \colo{lesquelles} ils se sont séparés sont légitimes. 
\item Elle a un sac à dos dans \colo{lequel} elle garde son téléphone.
\item Les deux jeunes femmes entre \colo{lequel} je me trouvais se ressemblaient énormément. 
\item La trottinette sur \colo{laquelle} j’ai fait des réparations est encore en panne. 
\item Les chaussures avec \colo{lesquelles} tu es venue était magnifique. 
\item J’ai perdu le cahier sur \colo{lequel} j’avais écrit son adresse mail. 
\item Les peluches (f) avec \colo{lesquelles} elle joue sont neuves. \item Je repeins le muret sur \colo{lequel} il avait tagué.
\item Les diplômées (f) parmi \colo{lesquelles} je me suis retrouvé avaient beaucoup de choses à m’apprendre. 
\item Ces dernières mois pendant \colo{lesquels} j’ai déménagé étaient intenses.
\end{enumerate}


\subsection{Complétez les phrases suivantes en remplaçant « de » par les pronoms relatifs : duquel, de laquelle, desquels ou desquelles.}

\begin{enumerate}
 \item Voici le navire à bord \colo{duquel} j’ai traversé les océans. \item Pose ton sac sur la table au-dessous de \colo{laquelle} il y a un chien. 
 \item Les chemins (m) le long \colo{desquelles} nous avons couru étaient en bitume. 
 \item C’est un livre à la fin \colo{duquel} tout finit mal. 
 \item La piscine près de \colo{laquelle} nous résidons est très sale. 
 \item Les filles (f) en face \colo{desquelles} nous habitons sont sénégalaises. 
 \item C’est un jardin au centre \colo{duquel} il y a des arbres centenaires.
 \item Il y a eu des élections (f) à la suite  \colo{desquelles} un nouveau président a été élu. 
 \item Je vous ai donné un document au bas \colo{duquel} j’ai apposé ma signature. 
 \item C’est un problème à propos de \colo{laquelle} nous sommes d’accord.
\end{enumerate}

\subsection{Complétez les phrases suivantes en utilisant les pronoms relatifs : auquel, à laquelle, auxquels ou auxquelles.} 
\begin{enumerate}
\item Dans la vie, il y a beaucoup de personnes (f) \colo{lesquelles} on s’attache. (tenir à qqch) 
\item La société (f) \colo{dans laquelle} vous avez consacré votre vie est en cessation d’activité. (se consacrer à faire qqch)
\item Les modules (m) \colo{sur lesquels} vous faites références sont fondamentaux. (faire référence à qqch)
\item Ce n’est pas une prévision \colo{à laquelle} je crois (croire à qqch) 
\item Voici les avantages (m) \colo{auxquelles} les salariés ont droit (avoir droit à qqch) 
\item Le projet \colo{auquel} elle pense tout le temps est réalisable. (penser à qqn) 
\item Ce principe \colo{auquel} ils étaient attaché n’est plus valable désormais. (être attaché à qqch) 
\item L’excursion \colo{à laquelle} je songeais nécessite de l’entraînement. (songer à qqch) 
\item C’est une mission \colo{à laquelle} je voudrais mettre fin. (mettre fin à qqch) 
\item La montre \colo{à laquelle} ils tenaient est cassée. (tenir à qqch)
\end{enumerate}
\subsection{Complétez les phrases suivantes en utilisant le pronom relatif qui convient.} 
\begin{enumerate}
\item La formation à \colo{laquelle} nous devions aller a été annulée. \item Où est le sac dans \colo{lequel} j’avais rangé mes affaires ? \item Voici le stylo avec \colo{lequel} tu pourras prendre des notes. \item La décision à \colo{laquelle} il s’est résigné la rend triste.
\item Les étagères sur \colo{lesquelles} j’ai placé des dossiers sont pratiques. 
\item Les voisines loin \colo{desquelles} j’étais placé parlaient du village. 
\item C’est une tendance à \colo{laquelle} il faut se résigner. 
\item Les amies chez \colo{lesquelles} nous dînerons sont généreuses. \item Les pratiques sportives \colo{auxquelles} elle s’adonne sont éreintantes. 
\item Ce chemin mène à un quartier au bout \colo{duquel} il y a une épicerie. 
\item Les activités (f) à \colo{laquelle} il a consacré son année sont utiles. 
\item Voici les fleurs (f) et les animaux (m) parmi \colo{lesquelles} nous avons vécu pendant six mois. 
\item Cette odeur (f) à \colo{laquelle} je ne pouvais m’habituer a disparu. 
\item La pièce dans \colo{laquelle} vous étudiez est climatisée. 
\item La matière à \colo{laquelle} elle s’intéresse est complexe. \item Nettoie le portail sur \colo{lequel} il y a de nombreuses saletés. 
\item La série devant \colo{laquelle} j’ai dormi était vraiment nulle. \item Le club de badminton \colo{auquel} nous avions adhéré n’existe plus. 
\item Les passe-temps (m) \colo{auxquels} je m’intéresse sont confidentiels. 
\item Les gens avec \colo{lesquels} j’étudie sont passionnants.
\end{enumerate}
\end{document}