\documentclass[12pt]{article}
\usepackage[margin=10mm]{geometry}
\usepackage[T1]{fontenc}
\usepackage[utf8]{inputenc}
\usepackage[french]{babel}
\usepackage{array}
\usepackage{hyperref}
\usepackage{pifont}
\usepackage{colortbl}
\usepackage{color}
\newcommand{\und}[1]{\underline{#1}}
\newcommand{\colo}[1]{{\color{blue}\textbf{#1}}}

\author{Jean Lucien RANDRIANANTENANA}
\title{Vers un français
impeccable (B2)\\SAYNA-FLE-PROJET1-052022}
\begin{document}
\maketitle
\tableofcontents
\newpage
\section{Les connecteurs logiques}
\subsection{Identifiez la relation logique}
\begin{enumerate}
	\item Prenez ce GPS dans le cas où vous vous perdriez. \colo{(Condition)}
	\item Elle ne parle que de son chien.
	\item Je ne portais pas d’imperméable, c'est pourquoi je suis mouillé. \colo{(Justification/Conséquence)}
	\item Il fait très chaud depuis le début de l’année, notamment ces derniers jours. \colo{(Illustration)}
	\item J'aime tous les légumes en dehors des navets.\colo{(Restriction)}
	\item Je partirai tôt demain matin. En effet, j'ai beaucoup de travail qui m’attend. \colo{(Justification)}
	\item Il ne parle pas anglais pourtant il est né à Londres. \colo{(Opposition)}
	\item À supposer que vous vouliez venir avec moi, je n'ai pas assez de place dans le van.\colo{(Supposition)}
\end{enumerate}
\subsection{INDICATIF OU SUBJONCTIF : Conjuguez les verbes à la forme qui convient}
\begin{enumerate}
	\item Tu es rousse alors que ton frère \colo{est} blond.
	\item Bien qu’il \colo{soit} parti en retard, il est arrivé en avance.
	\item Afin que tu \colo{comprenne} la problématique, je vais te l’expliquer. \item Ahmed \colo{est} plus grand que Fatou.
	\item Sa mère sera rassurée à condition qu'il \colo{suive} des cours particuliers.
	\item J’ai beaucoup étudié, c’est ainsi que je \colo{suis} devenu avocat.
	\item Donne-lui cette casquette qu‘il n’ \colo{ait} pas trop chaud à la tête.
	\item Après que chacun \colo{aie} gagné son siège, la séance peut commencer.
\end{enumerate}
\subsection{Choisissez le bon connecteur logique : }
\begin{enumerate}
	\item \colo{Puisque} tu as oublié ton argent, voici 10 euros.
	\item \colo{Comme} il était souffrant, il est retourné chez lui.
	\item Ce pays est très pauvre \colo{en revanche} ses habitants sont très gentils.
	\item J’habite à Antananarivo \colo{depuis} 7 ans.
	\item J’ai voté pour elle \colo{même si} je ne l’apprécie pas.
	\item Il est arrivé régulièrement en retard au travail \colo{par conséquent} il a été renvoyé.
	\item Je vais à la plage cet après-midi, \colo{pourvu qu’} il ne pleuve pas. \item Vous êtes allées dehors \colo{malgré} l’averse.
\end{enumerate}

\subsection{Terminez librement les phrases en utilisant un connecteur logique.}
\begin{enumerate}
	\item Il était absent à la réunion \colo{parce qu'il était malade}
	\item Cette société ne fait pas assez de bénéfice \colo{alors il s'en dette pour payer le salaire des employer}
	\item Elles se sont rendues au marché \colo{pour acheter des légumes}
	\item Vous avez pris le bus \colo{pour aller à la gare puis le train pour aller à Manakara}
	\item Tu n’as pas remporté la victoire, \colo{cependant tu a fait de bon progrès.}
\end{enumerate}

\subsection{Ces connecteurs utilisent-ils l'indicatif ou le subjonctif ?}
\begin{enumerate}
	\item Du fait que
	\item Pour que
	\item Alors que
	\item Jusqu'à ce que
	\item Bien que :\colo{Subjonctif}
	\item De sorte que
	\item Après que
\end{enumerate}

\section{La condition et l’hypothèse}
\subsection{Conjuguez le verbe au présent (conditionnel, subjonctif ou indicatif)}
\begin{enumerate}

	\item Je vous permets de sortir si vous me \colo{promettiez} de revenir dans une heure. (promettre) \\
	      \ding{212} Je vous permets de sortir à condition que vous me ------------ de revenir dans une heure.
	\item Si vous ------------ quelques progrès, vos parents seront contents. (faire) \\
	      \ding{212} Pour peu que vous ------------ quelques progrès, vos parents seront contents. \item Venez au concert, si vous ne ------------ pas vous reposer. (préférer) \\
	      \ding{212} Venez au concert, à moins que vous ne ------------ vous reposer.
	\item Si vous ------------ votre examen, que ferez-vous ? (rater) \\
	      \ding{212} Au cas où vous ------------ votre examen, que ferez-vous ? \\ \ding{212} En admettant que vous ------------votre examen, que ferez-vous ? \item Vous gagnerez le concours si vous ------------ (s'entraîner) \\ \ding{212} Vous gagnerez le concours à condition que vous ------------ \\ \ding{212} Vous gagnerez le concours pour peu que vous ------------
	\item Vous ne vous servirez du dictionnaire que si le professeur vous le ------------ (permettre) \\
	      \ding{212} Vous ne vous servirez pas du dictionnaire, à moins que le professeur vous le ------------
	\item Tu iras chez le dentiste, si ça te ------------ ou si ça ne te ------------ pas. (plaire) \\
	      \ding{212} Tu iras chez le dentiste, que ça te ------------ ou non.
\end{enumerate}
\subsection{Choisissez la bonne réponse}
\begin{enumerate}
		\item Dès que vous aurez terminé, nous (partir)\\ 
	      \ding{212} nous partirions \\ 
	      \ding{212} nous partirons 
	      \item Si vous étiez d'accord, nous (commencer) les travaux demain \\ \ding{212} nous commencerons \\ 
	      \ding{212} nous commencerions 
	      \item Je (vendre) ma voiture le mois prochain \\ 
	      \ding{212} je vendrai \\ 
	      \ding{212} je vendrais 
	      \item J'(écrire) bien cette lettre, mais je crains de faire des fautes \\ \ding{212} j'écrirai \\ 
	      \ding{212} j'écrirais 
	      \item Petite annonce : (prendre) des enfants en garde \\ 
	      \ding{212} (je) prendrai \\ 
	      \ding{212} (je) prendrais 
	      \item Autre annonce : (donner) leçons de français et de karaté \\ \ding{212} (je) donnerai \\ \ding{212} (je) donnerais 
	      \item Je (pouvoir) vous aider si je ne devais pas m'absenter \\ \ding{212} je pourrai \\ \ding{212} je pourrais.
	      \item Je (tâcher) de vous aider dès mon retour \\ 
	      \ding{212} je tâcherai \\ \ding{212} je tâcherais 
	      \item Il (vouloir) bien savoir comment faire. \\ 
	      \ding{212} "il (ou elle) voudra \\ 
	      \ding{212} "il (ou elle) voudrait" 
	      \item Je (répondre) sans faute \\ 
	      \ding{212} répondrai \\ 
	      \ding{212} répondrais
\end{enumerate}
\section{Les pronoms relatifs composés} 
\subsection{Complétez les phrases suivantes en utilisant les pronoms relatifs : lequel, laquelle, lesquels ou lesquelles.} 
\begin{enumerate}
\item Les raisons (f) pour ------------ ils se sont séparés sont légitimes. \item Elle a un sac à dos dans ------------ elle garde son téléphone. \item Les deux jeunes femmes entre ------------ je me trouvais se ressemblaient énormément. \item La trottinette sur ------------ j’ai fait des réparations est encore en panne. \item Les chaussures avec ------------ tu es venue était magnifique. \item J’ai perdu le cahier sur ------------ j’avais écrit son adresse mail. \item Les peluches (f) avec ------------ elle joue sont neuves. \item Je repeins le muret sur ------------ il avait tagué. \item Les diplômées (f) parmi ------------ je me suis retrouvé avaient beaucoup de choses à m’apprendre. \item Ces dernières mois pendant ------------ j’ai déménagé étaient intenses.
\end{enumerate}


\subsection{Complétez les phrases suivantes en remplaçant « de » par les pronoms relatifs : duquel, de laquelle, desquels ou desquelles.}

\begin{enumerate}
 \item Voici le navire à bord de ------------ j’ai traversé les océans. \item Pose ton sac sur la table au-dessous de ------------ il y a un chien. \item Les chemins (m) le long de ------------ nous avons couru étaient en bitume. \item C’est un livre à la fin de ------------ tout finit mal. \item La piscine près de ------------ nous résidons est très sale. \item Les filles (f) en face de ------------ nous habitons sont sénégalaises. \item C’est un jardin au centre de ------------ il y a des arbres centenaires.
 \item Il y a eu des élections (f) à la suite de ------------ un nouveau président a été élu. \item Je vous ai donné un document au bas de ------------ j’ai apposé ma signature. \item C’est un problème à propos de ------------ nous sommes d’accord.
\end{enumerate}

\subsection{Complétez les phrases suivantes en utilisant les pronoms relatifs : auquel, à laquelle, auxquels ou auxquelles.} 
\begin{enumerate}
\item Dans la vie, il y a beaucoup de personnes (f) ------------ on s’attache. (tenir à qqch) \item La société (f) ------------ vous avez consacré votre vie est en cessation d’activité. (se consacrer à faire qqch) \item Les modules (m) ------------ vous faites références sont fondamentaux. (faire référence à qqch) \item Ce n’est pas une prévision ------------ je crois (croire à qqch) \item Voici les avantages (m) ------------ les salariés ont droit (avoir droit à qqch) \item Le projet ------------ elle pense tout le temps est réalisable. (penser à qqn) \item Ce principe ------------ ils étaient attaché n’est plus valable désormais. (être attaché à qqch) \item L’excursion ------------ je songeais nécessite de l’entraînement. (songer à qqch) \item C’est une mission ------------ je voudrais mettre fin. (mettre fin à qqch) \item La montre ------------ ils tenaient est cassée. (tenir à qqch)
\end{enumerate}
\subsection{Complétez les phrases suivantes en utilisant le pronom relatif qui convient.} 
\begin{enumerate}
\item La formation à ------------ nous devions aller a été annulée. \item Où est le sac dans ------------ j’avais rangé mes affaires ? \item Voici le stylo avec ------------ tu pourras prendre des notes. \item La décision à ------------ il s’est résigné la rend triste.
\item Les étagères sur ------------ j’ai placé des dossiers sont pratiques. \item Les voisines loin de ------------ j’étais placé parlaient du village. \item C’est une tendance à ------------ il faut se résigner. \item Les amies chez ------------ nous dînerons sont généreuses. \item Les pratiques sportives à ------------ elle s’adonne sont éreintantes. \item Ce chemin mène à un quartier au bout de ------------ il y a une épicerie. \item Les activités (f) à ------------ il a consacré son année sont utiles. \item Voici les fleurs (f) et les animaux (m) parmi ------------ nous avons vécu pendant six mois. \item Cette odeur (f) à ------------ je ne pouvais m’habituer a disparu. \item La pièce dans ------------ vous étudiez est climatisée. \item La matière à ------------ elle s’intéresse est complexe. \item Nettoie le portail sur ------------ il y a de nombreuses saletés. \item La série devant ------------ j’ai dormi était vraiment nulle. \item Le club de badminton à ------------ nous avions adhéré n’existe plus. \item Les passe-temps (m) à ------------ je m’intéresse sont confidentiels. \item Les gens avec ------------ j’étudie sont passionnants.
\end{enumerate}
\end{document}